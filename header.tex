%scrhack, um @addtolists warnings zu unterdrücken, bei Paketen, die KOMA kennt (insbesondere listings)
\usepackage{scrhack}
\usepackage[T1]{fontenc}
\usepackage[utf8]{inputenc}
\usepackage[english]{babel}
\usepackage[scaled=1]{uarial} % set default font to arial
\renewcommand{\familydefault}{\sfdefault}
\usepackage{multirow}
\usepackage{float}
\usepackage{csquotes}

% enable hyperlinks in pdf documennt
\usepackage[hidelinks]{hyperref}
\usepackage[colorinlistoftodos]{todonotes}\reversemarginpar

%enable \gls
\usepackage[automake,nonumberlist,acronym,toc,nopostdot,nomain,numberedsection=autolabel]{glossaries}
\newglossary[slg]{symbolslist}{syi}{syg}{Symbols} % create add. symbolslist
\makeglossaries
% Displays unit in symbole list
\newglossarystyle{symbunitlong}{
	\setglossarystyle{long4col}
	\renewcommand*{\glossentry}[2]{
		\glstarget{##1}{\glossentryname{##1}} %
		& \glssymbol{##1}
		& \glossentrydesc{##1}  \tabularnewline
	}
}
\makeindex
\setacronymstyle{long-short}
\glsaddall

%% the following commands are needed for some matlab2tikz features
\usetikzlibrary{plotmarks}
\usetikzlibrary{arrows.meta}

\usepackage{graphicx}
\usepackage{tkz-euclide}
\usepackage{pgfplots}
\usepackage{tikzscale}
\pgfplotsset{
    every tick label/.append style = {/pgf/number format/assume math mode=true,font=\sffamily},
}
\pgfplotsset{max space between ticks=50}
\pgfplotsset{compat=newest}


%\usepackage{grffile}
\usepackage{amsmath}
\usepackage{listings}
\usepackage{rotating}
\usepackage{cleveref}
%\crefname{equation}{eq.}{eqs.}
%\Crefname{equation}{Eq.}{Eqs.}
\crefformat{equation}{Eq.~#2#1#3}
\Crefformat{equation}{Eq.~#2#1#3}
%\crefformat{subfigure}{Fig.~#2(#1)#3}
\crefname{table}{Fig.}{Figs.}
\crefname{figure}{Fig.}{Figs.}
\Crefname{table}{Fig.}{Figs.}
\crefname{chapter}{Chapter}{Chapters}
\Crefname{chapter}{Chapter}{Chapters}
\crefname{appendix}{Appendix}{}
\crefname{section}{Chapter}{Chapters}
\Crefname{section}{Chapter}{Chapters}
\crefname{subsection}{Chapter}{Chapters}
\Crefname{subsection}{Chapter}{Chapters}
\crefname{subsubsection}{Chapter}{Chapters}
\Crefname{subsubsection}{Chapter}{Chapters}

\usepackage{array}
\usepackage{booktabs}
\usepackage{color}
\usepackage{tensor}
\usepackage{pdfpages}
\usepackage{bm}
\usepackage[section]{placeins}

% set page margins etc.
\usepackage[
	headheight=1.25cm,
	headsep=1.25cm,
	left=3.0cm,
	right=2.0cm,
	top=3.0cm,
	bottom=2.5cm]{geometry}
	
\usepackage[labelformat=simple]{subcaption}
\usepackage[onehalfspacing]{setspace}
\AtBeginEnvironment{tabular}{\onehalfspacing}

\usepackage{amsmath}
\usepackage{amssymb}
\usepackage{siunitx}
\usepackage{tabto}
\usepackage{pbox}

%%%%%%%%%%%%%%%%%%%%%%%%%
% IKA - format	        %
%%%%%%%%%%%%%%%%%%%%%%%%%

%-----------------------Inhaltsverzeichnis---------------------------%
%Maximale Gliederungstiefe, die noch ins Inhaltsverzeichnis aufgenommen wird
%maximum depth for the table of contents
%Tiefe von 4 erlaubt, aber häufig Verwendung von 3
%\setcounter{tocdepth}{3}
\setcounter{tocdepth}{4}
%Nummerierung der Überschriften bis zu welcher Schachtelung
\setcounter{secnumdepth}{4}
% Formatierung des Inhaltsverzeichnisses
\setkomafont{disposition}{\normalfont} 
\addtokomafont{chapterentry}{\normalfont}
%COUNTER needed, to have 12pt space before AND after each new chapter, but 6pt else
%LaTeX lets us only set space before a contentsline, so we have to check, if a
%section is the first in its chapter, and only then add 12pt before the section
\newcounter{sectionref}
\setcounter{sectionref}{1}
%set indentation in table of contents
\makeatletter
\renewcommand{\l@chapter}{\vspace{12pt} \setcounter{sectionref}{1} \@dottedtocline{1}{0cm}{0.9cm}}
\renewcommand\l@section{ \ifthenelse{\value{sectionref}=1}{\vspace{12pt}}{\vspace{6pt}} \stepcounter{sectionref} \@dottedtocline{2}{0.3cm}{1.25cm}}
\renewcommand\l@subsection{\vspace{6pt}\@dottedtocline{3}{0.6cm}{1.34cm}}
\renewcommand*\l@subsubsection{\vspace{6pt}\@dottedtocline{4}{0.9cm}{1,82cm}}
%narrow dots, like in WORD
\renewcommand\@dotsep{1}% default is 4.5
\makeatother


%% labelin
%\newcommand{\reffig}[1]{Fig.~\ref{#1}}
%\newcommand{\refsec}[1]{Chapter~\ref{#1}}
%\newcommand{\refapp}[1]{Appendix~\ref{#1}}
%\newcommand{\refeq}[1]{Eq.~\ref{#1}}

% Anpassung der Gleichungsbeschriftung an das IKA-Layout, e.g. "Gl. 1-1"
\makeatletter
	\def\@eqnnum{{\normalfont \normalcolor Eq.\quad \theequation}}
	\def\tagform@#1{\maketag@@@{Eq.\quad\ignorespaces#1\unskip\@@italiccorr}}
\makeatother

% Anpassung der Tabellen- und Bildbeschriftung an das IKA-Layout, e.g. "Abb. 1-1:	" (Teil 2)
\renewcaptionname{english}{\figurename}{Fig.}
\renewcaptionname{english}{\tablename}{Fig.}
\renewcommand{\thefigure}{{\thechapter-\arabic{figure}}}
\renewcommand{\thetable}{{\thechapter-\arabic{figure}}}
\renewcommand{\theequation}{{\thechapter-\arabic{equation}}}

% Durchgehende Nummerierung für Abbildungen und Tabellen: IKA fordert Tabllen mit Abb-Caption: Abb. X-Y
\makeatletter
\let\c@table\c@figure
\makeatother





% Anpassung der Beschriftungen
\usepackage[
	style=base,
	font=onehalfspacing,
	format=hang,
	margin=0cm,
	singlelinecheck=false,
	skip=6pt
	]{caption}

% Anpassung der Absätze vor und nach einer figure Umgebung
\setlength{\intextsep}{12pt}

\renewcommand{\thesubfigure}{(\roman{subfigure})}

% RWTH Farbdefinitionen
\definecolor{blueRWTH}{cmyk}{1.0,0.5,0,0}
\definecolor{blueLightRWTH}{cmyk}{0.75,0.38,0,0}
\definecolor{blueLighterRWTH}{cmyk}{0.45,0.14,0,0}

\definecolor{greenRWTH}{cmyk}{0.7,0.0,1.0,0}
\definecolor{greenLightRWTH}{cmyk}{0.525,0,0.75,0}
\definecolor{greenLighterRWTH}{cmyk}{0.35,0,0.5,0}

\definecolor{orangeRWTH}{cmyk}{0,0.4,1.0,0}
\definecolor{orangeLightRWTH}{cmyk}{0,0.3,0.75,0}
\definecolor{orangeLighterRWTH}{cmyk}{0,0.2,0.5,0}

\definecolor{redRWTH}{cmyk}{0.15,1,1,0}
\definecolor{redLightRWTH}{cmyk}{0.1125,0.75,0.75,0}
\definecolor{redLighterRWTH}{cmyk}{0.075,0.5,0.5,0}
\definecolor{rwthLight}{cmyk}{0.0,0.0,0.0,0.5}


% Kopf- und Fußzeile
%%%%%%%%%%%%%%%%%%%%%%
%\usepackage[
%	headsepline=0.75pt,
%	plainheadsepline=true,
%	footsepline=false,
%	plainfootsepline
%	]{scrlayer-scrpage} % Für die Kopf- und Fußzeile inkl. Trennlinie und Titel groß geschrieben
%

%
%\ihead*[\thechapter \tabto{1cm}{\leftmark}]{\thechapter \tabto{1cm}{\leftmark}}
%\cohead*[]{}
%\rohead*[\thepage]{\thepage}
%\ifoot*[]{}
%\cofoot*[]{}
%\rofoot*[]{}
\usepackage{fancyhdr}

\pagestyle{fancy}
\renewcommand{\chaptermark}[1]{\markboth{#1}{}}

% Change the color of the header to the light gray and remove the spacing between header text and line
\let\oldheadrule\headrule% Copy \headrule into \oldheadrule
\renewcommand{\headrule}{\color{rwthLight}\oldheadrule}% Add colour to \headrule
\renewcommand{\headrulewidth}{0.75pt}

\fancypagestyle{ika}{%
    \fancyhf{}
    \lhead{\color{rwthLight}\thechapter \tabto{1cm} \leftmark}
    \rhead{\color{rwthLight}\thepage}
    \fancyfoot[]{}
    \rfoot{\color{rwthLight}\tiny{\thesisNo}}
}
\fancypagestyle{plain}{%
    \fancyhf{}
    \lhead{\color{rwthLight}\leftmark}
    \rhead{\color{rwthLight}\thepage}
    \fancyfoot[]{}
}
\fancypagestyle{apx}{%
    \fancyhf{}
    \lhead{\color{rwthLight}Appendix - \leftmark}
    \rhead{\color{rwthLight}\thepage}
    \fancyfoot[]{}
}


% XML Listing
\definecolor{gray}{rgb}{0.4,0.4,0.4}
\definecolor{darkblue}{rgb}{0.0,0.0,0.6}
\definecolor{cyan}{rgb}{0.0,0.6,0.6}

\lstset{
    basicstyle=\ttfamily,
    columns=fullflexible,
    frame = single, 
    numbers = left,
    showstringspaces=false,
    commentstyle=\color{gray}\upshape
}

\lstdefinelanguage{XML}
{
    morestring=[b]",
    morestring=[s]{>}{<},
    morecomment=[s]{<?}{?>},
    stringstyle=\color{black},
    identifierstyle=\color{darkblue},
    keywordstyle=\color{cyan},
    morekeywords={}% list your attributes here
}
\pgfkeys{/pgf/number format/.cd,1000 sep={}}
%%%%%%%%%%%%%%%%%%%%%%%%%%%%%%%%%%
% Formatierung der Überschriften
%%%%%%%%%%%%%%%%%%%%%%%%%%%%%%%%%%

% Kapitelüberschrift
\RedeclareSectionCommands[
  afterskip=4pt,
  beforeskip=4pt
]{chapter}

% Unterkapitel
\RedeclareSectionCommands[
  afterskip=4pt,
  beforeskip=4pt
]{section}

\RedeclareSectionCommands[
  afterskip=4pt,
  beforeskip=4pt
]{subsection}

% Unterkapitel
\RedeclareSectionCommands[
  afterskip=4pt,
  beforeskip=4pt
]{subsubsection}

\RedeclareSectionCommands[
  tocindent=0.3cm
]{section}

\RedeclareSectionCommands[
  tocindent=0.6cm
]{subsection}

\RedeclareSectionCommands[
  tocindent=0.9cm
]{subsubsection}

\RedeclareSectionCommands[
  tocbeforeskip=5pt
]{section,subsection,subsubsection}

% Tabstopp für Überschriften auf 1cm setzen
\renewcommand*{\chapterformat}{%
	\makebox[1cm][l]{\chapappifchapterprefix{\nobreakspace}\thechapter
	\IfUsePrefixLine{}}}
\renewcommand*{\sectionformat}{%
	\makebox[2cm][l]{\thesection}}
\renewcommand*{\subsectionformat}{%
	\makebox[2cm][l]{\thesubsection}}
\renewcommand*{\subsubsectionformat}{%
	\makebox[2cm][l]{\thesubsubsection}}

% Schriftarten
\setkomafont{chapter}{\normalfont\bfseries}
\setkomafont{section}{\normalfont\bfseries}
\setkomafont{subsection}{\normalfont\bfseries}
\setkomafont{subsubsection}{\normalfont\bfseries}
\setkomafont{pagehead}{\normalfont}
\setkomafont{chapterentry}{\normalfont}
