% !TeX root = ../main.tex
\chapter{Project}
\label{ch:pj}
For addressing the objectives of the thesis, a software application has been developed which helps us to understand the reasoning of the SpatialDETR model. Trying to make it adaptable, a model selection is allowed. Therefore, it is possible to use it for various SpatialDETR configuration, e.g with only query center projection. 
At the moment, the application is addressed to developers who want to see how the 3D Detection works behind the scene. 
It is possible to select the attention visualization mechanism (e.g head fusion, attention rollout, gradient rollout), the attention discard ratio, the  layer (6 in SpatialDETR), the camera (6 in NuScenes dataset). 
Furthemore, it is possible to select the prediction threshold for detection and to visualize the ground trouth bounding boxes. All this flexibility allows for a better understanding of the Transformer model and will help to decide, for example, which attention visualization mechanism better explains the model. 
Therefore this application can be adapted to normal users and authorities by selecting the best configuration for XAI.
